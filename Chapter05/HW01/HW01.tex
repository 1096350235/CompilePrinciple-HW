%        File: homework.tex
%     Created: 一 5月 07 01:00 下午 2018 C
% Last Change: 一 5月 07 01:00 下午 2018 C
%
\documentclass[UTF8,noindent]{ctexart}
\usepackage[a4paper,left=2.0cm,right=2.0cm,top=2.0cm,bottom=2.0cm]{geometry}
\usepackage{hyperref}
\usepackage{url}
\usepackage{graphicx}
\usepackage{amsmath}
\usepackage{amssymb}
\usepackage{enumitem}
\usepackage{tikz}
\usepackage{float}
\usepackage{listings}
\usepackage{xcolor}
\usepackage{forest}
\lstset{language = c,numbers=left, keywordstyle= \color{ blue!70 },commentstyle=\color{red!50!green!50!blue!50}, frame=shadowbox, rulesepcolor= \color{ red!20!green!20!blue!20 } 
} 
\usetikzlibrary{graphs}
\title{$Chapter\ 5-HW01$}
\author{$2015K8009929049$\ 冯吕}
\date{\today}
\begin{document}
\maketitle
\zihao{5}
\CJKfamily{zhsong}
$5.1.1$解:$(3+4)*(5+6)$对应的注释语法分析树如下:
\begin{figure}[H]
  \centering
  \begin{forest}
	[
	  {$L.val = 77$}, [
		{$E.val = 77$}[
		  {$T.val = 77$}[
			[
			  {$T.val = 7$}
			  % TODO
			  [
				{$F.val = 7$}
				[
				  {$($}
				]
				[
				  {$E.val = 7$}
				  % TODO
				  [
					{$E.val = 3$}
					[
					  {$T.val = 3$}
					  [
						{$F.val = 3$}
						[
						  {$digit.lexval = 3$}
						]
					  ]
					]
				  ]
				  [
					{$+$}
				  ]
				  [
					{$T.val = 4$}
					[
					  {$F.val = 4$}
					  [
						{$digit.lexval = 4$}
					  ]
					]
				  ]
				]
				[
				{$)$}
				]
			  ]
			]
			[
			  {$*$}
			]
			[
			  {$F.val=11$}
			  % TODO
			  [
				{$($}
			  ]
			  [
				{$E.val = 11$}
				% TODO
				[
				  {$E.val = 5$}
				  [
					{$T.val = 5$}
					[
					  {$F.val = 5$}
					  [
						{$digit.lexval = 5$}
					  ]
					]
				  ]
				]
				[
				  {$+$}
				]
				[
				  {$T.val = 6$}
				  [
					{$F.val = 6$}
					[
					  {$digit.lexval = 6$}
					]
				  ]
				]
			  ]
			  [
			  {$)$}
			  ]
			]
		  ]
		]
	  ]
	  [{$n$}]
	]
  \end{forest}
\end{figure}

$5.1.2$解:扩展后的$SSD$如下:
\begin{center}
  \begin{tabular}{|l|l|l|}
	\hline
	& 产生式& 语法规则\\
	\hline
  $1)$ & $L\rightarrow E\ n$ & $L.val = E.val$\\
  \hline
$2)$ & $E\rightarrow TE'$ & $E'.inh = T.val$ ,\ $E.val = E'. syn$ \\
\hline
$3)$ & $E'\rightarrow + TE_1'$ & $E_1'.inh = E'.inh + T.val$,\ $E'.syn = E_1'.syn$ \\
\hline
$4)$ & $E'\rightarrow \epsilon$ & $E'.syn = E'.inh$ \\
\hline
$5) $ & $T\rightarrow FT'$ & $T'.inh = F.val $,\ $T.val = T'.syn$\\
\hline 
$6)$ & $T'\rightarrow *FT_1'$ & $T_1'.inh = T'.inh * F.val$, \ $T'.syn = T_1'.syn$\\
\hline 
$7)$ & $T'\rightarrow \epsilon$ & $T'.syn = T'.inh$\\
\hline 
$8)$ & $F\rightarrow (E)$ & $F.val = E.val$\\
\hline 
$9)$ & $F\rightarrow digit$ & $F.val = digit.lexval$\\
\hline
  \end{tabular}
\end{center}

\newpage

$5.2.3$解:
\begin{itemize}
  \item 
$(1)$ $1$不满足,$2$不满足,$3$满足, $4$不满足;
\item
$(2)$ $1$满足,$2$满足,$3$满足, $4$不满足($B.i = A.s + C.s$);
\item $(3)$ 只要规则中不出现循环赋值,就存在。因此,$1$存在,$2$存在,$3$存在,而$4$中,$A.s = D.i = B.i + C.i = A.s+C.s+C.i$,所以不存在。
\end{itemize}


\end{document}


