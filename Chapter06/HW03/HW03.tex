%        File: HW.tex
%     Created: 五 6月 08 08:00 上午 2018 C
% Last Change: 五 6月 08 08:00 上午 2018 C
%
\documentclass[UTF8,noindent]{ctexart}
\usepackage[a4paper,left=2.0cm,right=2.0cm,top=2.0cm,bottom=2.0cm]{geometry}
\usepackage{hyperref}
\usepackage{url}
\usepackage{graphicx}
\usepackage{amsmath}
\usepackage{amssymb}
\usepackage{enumitem}
\usepackage{forest}
\usepackage{tikz}
\usepackage{float}
\usepackage{listings}
\usepackage{xcolor}
\lstset{language = c,numbers=left, keywordstyle= \color{ blue!70 },commentstyle=\color{red!50!green!50!blue!50}, frame=shadowbox, rulesepcolor= \color{ red!20!green!20!blue!20 } 
} 
\usetikzlibrary{graphs}
\title{$Chapter\ 6-HW03$}
\author{$2015K8009929049$\ 冯吕}
\date{\today}
\begin{document}
\maketitle
\zihao{5}
\CJKfamily{zhsong}
$6.6.1$解:
\begin{center}
  \begin{tabular}{|l|l|}
	\hline
	$Production$ & $Syntax\ Rule$\\
	\hline
	$S\rightarrow repeat\ S_1\ while\ B$ & $S_1.next = newlabel()$\\
	& $B.true = newlabel()$\\
	& $B.false = S.next$\\
	& $S.code = label(B.true) \parallel S_1.code$\\
	& $\parallel label(S_1.next) \parallel B.code $\\
	&\\
	$S\rightarrow for(S_1; B; S_2) \ S_3$ & $S_1.next = newlabel()$\\
	& $B.true = newlabel()$ \\
	& $B.false = S.next$\\
	& $S_2.next = S_1.next$\\
	& $S_3.next = newlabel()$ \\
	& $S.code = S_1.code$\\
	& $\parallel label(S_1.next) \parallel B.code$\\
	& $\parallel label(B.true) \parallel S_3.code$\\
	& $\parallel S_3.next\parallel S_2.code$\\
	& $\parallel gen('goto', S_1.next)$\\
	\hline
  \end{tabular}
\end{center}

$6.7.1$解:
\begin{align*}
  & 100\  \ \ if\ a = = b,\ goto\ \_\\
  & 101\ \ \ goto \ \_\\
  & 102 \ \ \ if \ c = = d,\  goto\ \_\\
  & 103 \ \ \ goto \ \_\\
  & 104 \ \ \ if \ e = = f ,\  goto\ \_\\
  & 105\ \ \ goto \ \_
\end{align*}
$1)$
各个子表达式的 $truelist$ 和 $falselist$ 如下图所示:
\begin{center}
\begin{forest}
  for tree= {math content}
  [{B.t = \{102, 104\}, B.f=\{101,105\}}
	[	{B.t = \{100\},B.f =\{101\}}
	  [{a}]
	  [{= =}]
	  [{b}]
	]
	[{\&\&}]
	[{M.i = 102}
	  [{\epsilon}]
	]
	[
{B.t=\{102,104\},B.f=\{105\}}
[{$($}]
	[{B.t=\{102,104\},B.f=\{105\}}
	  [{B.t=\{102\},B.f=\{103\}}
		[{c}]
		[{= =}]
		[{d}]
	  ]
	  [{\parallel}]
	  [{M.i = 104}
		[{\epsilon}]
	  ]
	  [{B.t=\{104\},B.f=\{105\}}
		[{e}]
		[{= =}]
		[{f}]
	  ]
	]
	[
	{$)$}
	]
]]
\end{forest}
\end{center}

分别用$L_1$和$L_2$ 表示表达式的真、假两个出口,则回填后的指令序列为:
\begin{align*}
  & 100\  \ \ if\ a = = b,\ goto\ 102\\
  & 101\ \ \ goto \ L_2\\
  & 102 \ \ \ if \ c = = d,\  goto\ L_1\\
  & 103 \ \ \ goto \ 104\\
  & 104 \ \ \ if \ e = = f ,\  goto\ L_1\\
  & 105\ \ \ goto \ L_2
\end{align*}

$2)$ 
各个子表达式的 $truelist$ 和 $falselist$ 如下图所示:
\begin{center}
  \begin{forest}
	for tree = {math content}
	[{B.t=\{100,102,104\}, B.f=\{105\}}
	  [
		{B.t =\{100,102\},B.f=\{103\}}
		[
		  {$($}
		]
	  [{B.t =\{100,102\},B.f=\{103\}}
		[{B.t=\{100\},B.f=\{101\}}
		  [{a}]
		  [{= =}]
		[{b}]
		]
		[{\parallel}]
		[{M.i = 102}
		  [{\epsilon}]
		]
		[{B.t=\{102\},B.f=\{103\}}
		  [{c}]
		  [{= =}]
		  [{d}]
		]
	  ]
	  [
	  {$)$}
	  ]
	]
	  [{\parallel}]
	  [{M.i = 104}
		[{\epsilon}]
	  ]
	  [{B.t=\{104\},B.f=\{105\}}
		[{e}]
		[{= =}]
		[{f}]
	  ]
	]
  \end{forest}
\end{center}

分别用$L_1$和$L_2$ 表示表达式的真、假两个出口,则回填后的指令序列为:
\begin{align*}
  & 100\  \ \ if\ a = = b,\ goto\ L_1\\
  & 101\ \ \ goto \ 102\\
  & 102 \ \ \ if \ c = = d,\  goto\ L_1\\
  & 103 \ \ \ goto \ 104\\
  & 104 \ \ \ if \ e = = f ,\  goto\ L_1\\
  & 105\ \ \ goto \ L_2
\end{align*}

$3)$ 
各个子表达式的 $truelist$ 和 $falselist$ 如下图所示:
\begin{center}
  \begin{forest}
	for tree = {math content}
	[{B.t=\{104\}, B.f=\{101,103,105\}}
	  [{B.t =\{102\},B.f=\{101,103\}}
		[
		  {$($}
		]
	  [{B.t =\{102\},B.f=\{101,103\}}
		[{B.t=\{100\},B.f=\{101\}}
		  [{a}]
		  [{= =}]
		[{b}]
		]
		[{\&\&}]
		[{M.i = 102}
		  [{\epsilon}]
		]
		[{B.t=\{102\},B.f=\{103\}}
		  [{c}]
		  [{= =}]
		  [{d}]
		]
	  ]
	  [
	  {$)$}
	  ]
	]
	  [{\&\&}]
	  [{M.i = 104}
		[{\epsilon}]
	  ]
	  [{B.t=\{104\},B.f=\{105\}}
		[{e}]
		[{= =}]
		[{f}]
	  ]
	]
  \end{forest}
\end{center}

分别用$L_1$和$L_2$ 表示表达式的真、假两个出口,则回填后的指令序列为:
\begin{align*}
  & 100\  \ \ if\ a = = b,\ goto\ 102\\
  & 101\ \ \ goto \ L_2\\
  & 102 \ \ \ if \ c = = d,\  goto\ 104\\
  & 103 \ \ \ goto \ L_2\\
  & 104 \ \ \ if \ e = = f ,\  goto\ L_1\\
  & 105\ \ \ goto \ L_2
\end{align*}
\end{document}
