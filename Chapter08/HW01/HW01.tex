%        File: HW.tex
%     Created: 日 7月 01 08:00 下午 2018 C
% Last Change: 日 7月 01 08:00 下午 2018 C
%
\documentclass[UTF8,noindent]{ctexart}
\usepackage[a4paper,left=2.0cm,right=2.0cm,top=2.0cm,bottom=2.0cm]{geometry}
\usepackage{hyperref}
\usepackage{url}
\usepackage{graphicx}
\usepackage{amsmath}
\usepackage{amssymb}
\usepackage{enumitem}
\usepackage{tikz}
\usepackage{float}
\usepackage{listings}
\usepackage{xcolor}
\lstset{language = c,numbers=left, keywordstyle= \color{ blue!70 },commentstyle=\color{red!50!green!50!blue!50}, frame=shadowbox, rulesepcolor= \color{ red!20!green!20!blue!20 } 
} 
\usetikzlibrary{graphs}
\title{$Chapter\ 8 -HW01$}
\author{$2015K8009929040$\ 冯吕}
\date{\today}
\begin{document}
\maketitle
\zihao{5}
\CJKfamily{zhsong}
%\section*{$homework\ 1$}

$8.2.5$ 解:下面语句序列生成的代码为:
\begin{align*}
&LD\ R_2,\ \#0\\
&LD\ R_1, \ R_2\\
&LD\ R_3, \ n\\
L_1: & \\
&SUB \ R_4, \ R_1,\ R_3\\
&BGTZ \ R_4, \ L_2\\
& ADD\ R_2,\ R_2,\ R_1\\
&ADD\ R_1,\ R_1,\ \#1\\
&BR\ L_1\\
L_2: &
\end{align*}

指令代价为$2+1+2+1+2+1+2+2 =13 $.

$8.3.3$解:使用栈式分配,生成的代码如下:
\begin{align*}
  & LD\ R_1,\ i\\
  & MUL\ R_1,\ R_1,\ 4\\
  & ADD\ R_1,\ R_1,\ SP\\
  & LD\ R_2,\ a(R_1)\\
  & ST\ x(SP),\ R_2\\
  &\\
  &LD\ R_3,\ j\\
  & MUL\ R_3,\ R_3, \ 4\\
  & ADD \ R_3,\ R_3,\ SP\\
  & LD\ R_4,\ b(R_3)\\
  & ST\ y(SP),\ R_4\\
  &\\
  &ST\ a(R_1),\ y(SP)\\
  & ST\ b(R_3),\ x(SP)
\end{align*}
\end{document}


