%        File: HW01.tex
%     Created: 一 7月 09 06:00 下午 2018 C
% Last Change: 一 7月 09 06:00 下午 2018 C
%
\documentclass[UTF8,noindent]{ctexart}
\usepackage[a4paper,left=2.0cm,right=2.0cm,top=2.0cm,bottom=2.0cm]{geometry}
\usepackage{hyperref}
\usepackage{url}
\usepackage{graphicx}
\usepackage{amsmath}
\usepackage{amssymb}
\usepackage{enumitem}
\usepackage{tikz}
\usepackage{float}
\usepackage{listings}
\usepackage{xcolor}
\lstset{language = c,numbers=left, keywordstyle= \color{ blue!70 },commentstyle=\color{red!50!green!50!blue!50}, frame=shadowbox, rulesepcolor= \color{ red!20!green!20!blue!20 } 
} 
\usetikzlibrary{graphs}
\title{$Chapter\ 1-HW01$}
\author{$2015K8009929049$\ 冯吕}
\date{\today}
\begin{document}
\maketitle
\zihao{5}
\CJKfamily{zhsong}
$1.1.2$ 解:
\begin{itemize}
  \item 编译器相对解释器的优点:编译器把源代码编译成目标代码,生成的程序执行时不再需要编译器。由编译器编译产生的目标程序通常运行速度比解释器要快很多。但由编译器产生的机器代码依赖于体系结构,移植到其他平台需要重新编译源程序;
	\item 解释器相对编译器的优点:解释器逐条语句的执行源程序,执行过程清晰直观,因此解释器的错误诊断效果通常比编译器要好。另外,依赖解释执行的程序可移植性一般更好,移植到其他平台时可直接执行或修改较少代码。
\end{itemize}

$1.1.3$ 解:汇编指令是机器指令的助记符,通常情况下,一条汇编指令就对应一条机器指令。生成汇编指令比直接生成机器语言会更方便进行调试、优化和输出。

$1.6.1$ 解:
\begin{lstlisting}
int w, x, y, Z;
int i = 4; int j = 5;
{
  int j = 7;
  i = 6;
  w = i + j;
}
x = i + j;
{
  int i = 8;
  y = i + j;
}
z = i + j;
\end{lstlisting}
\begin{itemize}
  \item 在第二行声明了变量 $i, j$,它们具有全局作用域。而在第一个大括号内,重新声明了一个变量$j$,因此,覆盖了前面声明的变量$j$的作用域,而$i$没有重新声明,所以,和前面的是同一个变量,只是重新赋值为了$6$,所以$w=i+j=6 + 7 = 13$;
  \item 离开第一个大括号之后,大括号内声明的$j$消失,所以$x = i +j = 6 + 5 = 11$;
  \item 进入第二个大括号之后,同理,重新声明的$i$覆盖了外面的$i$的作用域,所以$y = i + j = 8 + 5 = 13$;
  \item 离开大括号后,大括号内的$i$消失,$z = i + j = 6 + 5 = 11$。
\end{itemize}

$1.6.2$ 解:和上题同样分析,可知:
\begin{itemize}
  \item $w = i + j = 5 + 4 = 9$;
  \item $ x = i + j = 3 + 4 = 7$;
  \item $ y = i + j = 7 + 6 = 13$;
  \item $ z = i + j = 7 + 4 = 11$;
\end{itemize}

$1.6.4$ 解:
\begin{lstlisting}
#define a (x+1)
int x = 2;
void b() { 
  x = a;
  printf ( "%d\n", x ); 
  }
  void c() { 
	int x = 1; 
	printf ( "%d", a ); 
	}
	void main(){
	  b();
	  c():
	}
	\end{lstlisting}
	在函数外面,定义了一个宏,声明了一个$int$型变量$x$并初始化为$2$,所以,该变量具有全局作用域。在函数$b$中,没有重新声明$x$,因此$x = a = (x + 1) = 2 + 1= 3$;而在函数$c$内,重新声明了一个局部变量$x$并初始化为$1$,因此$a = (x + 1) = 1 + 1 = 2$。所以,打印结果为$3, 2$。

\end{document}


